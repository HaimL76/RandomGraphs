\documentclass{article}
\usepackage{amsfonts}
\usepackage{amsmath}
\usepackage{graphicx} % Required for inserting images

\title{RandomGraphs}
\author{haiml76 }
\date{August 2025}

\begin{document}

\maketitle

\section{}
\subsection{}
Let $G_0$ be a specific graph with $n$ vertices and $m_0$ edges, then $\mathbb{P}[G\sim{G(n,p)}=G_0]=p^{m_0}(1-p)^{\binom{n}{2}-m_0}\Rightarrow{\mathbb{P}[G\sim{G(n,p)}\in{A}]}=\mathbb{P}[\bigcup_{G_0\in{A}}G\sim{G(n,p)}=G_0]=\sum_{G_0\in{A}}p^{m_0}(1-p)^{\binom{n}{2}-m_0}$. But this is a polynomial in $p$, hence continous, and for every $0\leq{p\leq{q}}\leq{1}$. we have $\mathbb{P}[G\sim{G(n,p)}=G_0]=p^{m_0}(1-p)^{\binom{n}{2}-m_0}\leq{q^{m_0}(1-q)^{\binom{n}{2}-m_0}}=\mathbb{P}[G\sim{G(n,q)}=G_0]$, which applies also for the sum of probabilities over $\bigcup_{G_0\in{A}}G_0$. Thus, $f(p):=\mathbb{P}[G\sim{G(n,p)}\in{A}]$ is both continous and monotone increasing. We know that $\mathbb{P}[G\sim{G(n,0)}=\phi]=1\Rightarrow{\mathbb{P}[G\sim{G(n,0)}\in{A}]=0}$, and that $\mathbb{P}[G\sim{G(n,1)}\in{A}]=1$, hence $f(0)=0$ and $f(1)=1$, and by the intermediate value theorem, for each $0\leq{v}\leq{1}$, there must exist an argument $0\leq{p}\leq{1}$ s.t. $v=f(p)$. We choose $v=\frac{1}{2}$, hence there must exist some $0\leq{p^{\ast}}\leq{1}$ s.t. $v=f(p^{\ast})=\frac{1}{2}.$
\subsection{}
We check that $1-kp\leq{(1-p)^k}$, for all $k\in\mathbb{N}$, by induction.

For $k=1$ it is trivial, for $k+1$, we have $(1-p)^{k+1}=(1-p)^k(1-p)$, and by the assumption, $(1-p)^k(1-p)\geq{(1-kp)(1-p)}=1-kp-p+kp^2=1-(k+1)p+kp^2$, but $kp^2>0$, so $(1-p)^{k+1}=(1-p)^k(1-p)\geq{1-(k+1)p+kp^2}>1-(k+1)p$, which proves the assumption.

But it means that for each potential edge $e$, 

$\mathbb{P}[e\notin{G\sim{G(n,kp)}}]=1-kp\leq(1-p)^k=(\mathbb{P}[e\notin{G\sim{G(n,p)}}])^k=\mathbb{P}[e\notin{G\sim\bigcup_{i=1}^{k}G(n,p)}]\Rightarrow{\mathbb{P}[e\in{G\sim{G(n,kp)}}]\geq{\mathbb{P}[e\in{G\sim\bigcup_{i=1}^{k}G(n,p)]}}}$, 

but by a theorem coming from staged exposure, it means that if $A$ is an increasing monotone property, then $\mathbb{P}[G\sim{G(n,kp)}\in{A}]\geq{\mathbb{P}[G\sim\bigcup_{i=1}^{k}G(n,p)\in{A}]}\Rightarrow{\mathbb{P}[G\sim{G(n,kp)}\notin{A}]\leq{\mathbb{P}[G\sim\bigcup_{i=1}^{k}G(n,p)\notin{A}]}=(\mathbb{P}[G\sim{G(n,p)}\notin{A}]})^k$

\subsection{}
$\omega(n)\rightarrow{\infty}$

Let $n_k:=\min\{n\in\mathbb{N} : \lfloor\omega(n_k)\rfloor\geq{k}\}$, we are guaranteed to have such $n_k$ for every $k\geq{1}$, otherwise there exists some $k_0$ s.t. $\lfloor\omega(n)\rfloor\leq{k_0}$ for every $n\in\mathbb{N}$, in contradiction to $\omega(n)\rightarrow\infty$. 
Hence, for every $k\geq{1}$ we have $n_k\in\mathbb{N}$, s.t. $\mathbb{P}[G(n_k,\omega(n_k)p^{\ast})\notin{A}]\leq{\mathbb{P}[G(n_k,kp^{\ast})\notin{A}]}$, this is true because $\omega(n)\geq\lfloor\omega(n)\rfloor\geq{k}$ and $A$ is a monotone increasing property. But from 1.2 we know that $\mathbb{P}[G(n_k,kp^{\ast})\notin{A}]\leq{(\mathbb{P}[G(n_k,p^{\ast})\notin{A}])^k}$, but $\lim_{k\rightarrow\infty}(\mathbb{P}[G(n_k,p^{\ast})\notin{A}])^k=\lim_{k\rightarrow\infty}\frac{1}{2^k}=0\Rightarrow{\lim_{n\rightarrow\infty}\mathbb{P}[G(n,\omega{p^{\ast}})\in{A}]}=1$.

For $\mathbb{P}[G(n,\frac{p^{\ast}}{\omega(n)})\in{A}]$, we observe that $0\leq{p^{\ast}}\leq{1}\Rightarrow{\lim_{n\rightarrow\infty}\frac{p^{\ast}}{\omega(n)}}\leq{\lim_{n\rightarrow\infty}\frac{1}{\omega(n)}}=0$, as $\omega(n)$ tends to infinity.
But then $\lim_{n\rightarrow\infty}\mathbb{P}[G(n,\frac{p^{\ast}}{\omega(n)})\in{A}]=\lim_{p\rightarrow{0}}\mathbb{P}[G(n,p)\in{A}]=\mathbb{P}[G(n,0)\in{A}]$, as $\mathbb{P}[G(n,p)\in{A}]$ is continuous, but from 1.1 we know that $\mathbb{P}[G(n,0)\in{A}]=0$.
\end{document}
